% main.tex
\documentclass[11pt,a4paper]{report}
\usepackage[utf8]{inputenc}
\usepackage[T1]{fontenc}
\usepackage[french]{babel}
\usepackage{geometry}
\geometry{margin=2.2cm}
\usepackage{lmodern}
\usepackage{microtype}
\usepackage{hyperref}
\usepackage{graphicx}
\usepackage{float}
\usepackage{booktabs}
\usepackage{longtable}
\usepackage{caption}
\usepackage{subcaption}
\usepackage{tcolorbox}
\usepackage{listings}
\usepackage{pgfplots}
\usepackage{pgfplotstable}
\pgfplotsset{compat=1.18}
\usepackage{array}
\usepackage{multirow}

\hypersetup{
  pdftitle={Analyse de trafic réseau -- Rapport de projet},
  pdfauthor={Yacine Sehli},
  colorlinks=true,
  linkcolor=blue,
  citecolor=blue,
  urlcolor=blue
}

% Metadata
\title{\huge \textbf{Analyse de trafic réseau réalisée avec Wireshark} \\ [1cm] \large Rapport de projet technique}
\author{Auditeur : Yacine Sehli \\}
\date{\today}

\begin{document}
\maketitle


\tableofcontents
\clearpage

\chapter*{Résumé}
\addcontentsline{toc}{chapter}{Résumé}
Ce rapport présente une analyse technique et détaillée de trois captures réseau réalisées avec Wireshark : une capture \textit{loopback}, une capture \textit{Ethernet} et une capture \textit{Wi-Fi}. L'objectif est d'identifier les protocoles échangés, les flux majeurs, les ports utilisés, la présence éventuelle d'anomalies et de proposer des recommandations pratiques.\\

\textbf{Remarque importante :} Les graphiques et tableaux inclus dans ce document utilisent des jeux de données illustratifs. Dans l'annexe, des commandes \texttt{tshark} sont fournies pour extraire les données réelles depuis vos fichiers \texttt{.pcapng} et remplacer les jeux de données fictifs.

\chapter{Introduction}
\section{Contexte}
L'analyse du trafic réseau est essentielle pour : la sécurité, le diagnostic d'incidents, l'optimisation des performances et la validation du comportement d'applications. Les fichiers fournis représentent trois contextes distincts :
\begin{itemize}
  \item \textbf{Loopback} : trafic localhost (échanges inter-processus). Fichier : \texttt{adapteur for loopback.pcapng}.
  \item \textbf{Ethernet} : trafic filaire sur interface physique. Fichier : \texttt{wireshark1.pcapng}.
  \item \textbf{Wi-Fi} : trafic sans fil capturé sur interface Wi-Fi. Fichier : \texttt{wireshark2 wifi.pcapng}.
\end{itemize}

\section{Objectifs}
\begin{enumerate}
  \item Extraire les statistiques principales (nombre de paquets, durée, protocoles dominants).
  \item Identifier flux TCP/UDP significatifs (adresses IP et ports).
  \item Relever anomalies (ex. scans, retransmissions, paquets mal formés).
  \item Comparer les caractéristiques des trois interfaces.
  \item Formuler des recommandations.
\end{enumerate}

\chapter{Méthodologie}
\section{Outils utilisés}
\begin{itemize}
  \item \textbf{Wireshark} pour inspection visuelle et filtrage.
  \item \textbf{tshark} (ligne de commande) pour extraire des statistiques et générer des CSV.
\end{itemize}

\section*{Analyse par fichier}
% NOTE: Les tableaux et graphiques ci-dessous utilisent des données illustratives.
\section{Analyse générale — Loopback (\texttt{adapteur for loopback.pcapng})}
\subsection{Statistiques principales}
\begin{table}[H]
\centering
\caption{Statistiques synthétiques — Loopback}
\begin{tabular}{lrr}
\toprule
Mesure & Valeur (ex.) & Commentaire \\
\midrule
Nombre total de paquets & 1\,234 & capture courte, échanges locaux \\
Durée de la capture & 12.34 s & période observée \\
Débit moyen & 100 pkt/s & ordre de grandeur \\
\bottomrule
\end{tabular}
\end{table}

\subsection{Protocoles observés}
\begin{table}[H]
\centering
\caption{Distribution protocolaire — Loopback}
\begin{tabular}{lrr}
\toprule
Protocole & Paquets & Pourcentage \\
\midrule
TCP & 820 & 66.5\% \\
UDP & 150 & 12.2\% \\
DNS & 60 & 4.9\% \\
HTTP & 45 & 3.6\% \\
Autres (ARP, ICMP...) & 159 & 12.8\% \\
\bottomrule
\end{tabular}
\end{table}

\subsection{Flux et ports remarquables}
\begin{itemize}
  \item Connexions TCP locales sur 127.0.0.1 : ports 5000–5020 (trafic applicatif).
  \item Requêtes DNS locales vers résolveur local (UDP 53).
  \item Quelques échanges HTTP locaux (tests d'API).
\end{itemize}

\section{Analyse générale — Ethernet (\texttt{wireshark1.pcapng})}
\subsection{Statistiques principales}
\begin{table}[H]
\centering
\caption{Statistiques synthétiques — Ethernet}
\begin{tabular}{lrr}
\toprule
Mesure & Valeur (ex.) & Commentaire \\
\midrule
Nombre total de paquets & 12\,345 & capture plus importante \\
Durée de la capture & 300 s & 5 minutes \\
Débit moyen & 41 pkt/s & \\
\bottomrule
\end{tabular}
\end{table}

\subsection{Protocoles observés}
\begin{table}[H]
\centering
\caption{Distribution protocolaire — Ethernet}
\begin{tabular}{lrr}
\toprule
Protocole & Paquets & Pourcentage \\
\midrule
TCP & 6\,200 & 50.2\% \\
UDP & 3\,100 & 25.1\% \\
ARP & 1\,234 & 10.0\% \\
DNS & 567 & 4.6\% \\
HTTP/HTTPS & 876 & 7.1\% \\
\bottomrule
\end{tabular}
\end{table}

\subsection{Top Talkers (adresses IP)}
\begin{table}[H]
\centering
\caption{Top 5 adresses IP — Ethernet }
\begin{tabular}{lr}
\toprule
Adresse IP & Nombre de paquets \\
\midrule
192.168.1.10 & 3\,200 \\
192.168.1.1 & 2\,100 \\
93.184.216.34 & 1\,200 \\
172.217.14.78 & 900 \\
192.168.1.50 & 700 \\
\bottomrule
\end{tabular}
\end{table}

\subsection{Ports les plus utilisés}
\begin{table}[H]
\centering
\caption{Ports (TCP/UDP) — Ethernet}
\begin{tabular}{lr}
\toprule
Port & Nombre de flux \\
\midrule
80 (HTTP) & 420 \\
443 (HTTPS) & 1\,500 \\
53 (DNS) & 567 \\
22 (SSH) & 120 \\
123 (NTP) & 85 \\
\bottomrule
\end{tabular}
\end{table}

\section{Analyse générale — Wi-Fi (\texttt{wireshark2 wifi.pcapng})}
\subsection{Statistiques principales}
\begin{table}[H]
\centering
\caption{Statistiques synthétiques — Wi-Fi}
\begin{tabular}{lrr}
\toprule
Mesure & Valeur (ex.) & Commentaire \\
\midrule
Nombre total de paquets & 8\,765 & capture Wi-Fi (trafic mixte) \\
Durée de la capture & 600 s & 10 minutes \\
Débit moyen & 14.6 pkt/s & incl. management frames \\
\bottomrule
\end{tabular}
\end{table}

\subsection{Protocoles observés}
\begin{table}[H]
\centering
\caption{Distribution protocolaire — Wi-Fi}
\begin{tabular}{lrr}
\toprule
Protocole & Paquets & Pourcentage \\
\midrule
802.11 management & 2\,500 & 28.5\% \\
TCP & 3\,900 & 44.5\% \\
UDP & 1\,000 & 11.4\% \\
ARP & 200 & 2.3\% \\
DNS & 165 & 1.9\% \\
\bottomrule
\end{tabular}
\end{table}

\subsection{Observations Wi-Fi}
\begin{itemize}
  \item Présence de trames de management (beacon, probe) — normal en Wi-Fi.  
  \item Quelques retransmissions identifiées (signal instable ou collisions).
  \item Flux chiffrés (HTTPS) majoritaires vers IP externes.
\end{itemize}

\chapter{Visualisations}
% NOTE: Les graphiques ci-dessous utilisent des données d'exemple.
\section{Répartition protocolaire}
\begin{figure}[H]
  \centering
  \begin{tikzpicture}
    \begin{axis}[
      ybar stacked,
      width=0.9\textwidth,
      height=14cm,
      bar width=19pt,
      xlabel={Fichiers captures},
      ylabel={Nombre de paquets},
      symbolic x coords={Loopback, Ethernet, Wi-Fi},
      xtick=data,
      legend style={at={(0.5,-0.15)}, anchor=north,legend columns=-1},
      nodes near coords,
      nodes near coords align={vertical},
    ]
      % Data are illustrative; replace with real CSV via \addplot table
      \addplot+[ybar] coordinates {(Loopback,820) (Ethernet,6200) (Wi-Fi,3900)}; % TCP
      \addplot+[ybar] coordinates {(Loopback,150) (Ethernet,3100) (Wi-Fi,1000)}; % UDP
      \addplot+[ybar] coordinates {(Loopback,60) (Ethernet,567) (Wi-Fi,165)}; % DNS
      \addplot+[ybar] coordinates {(Loopback,45) (Ethernet,876) (Wi-Fi,0)}; % HTTP
      \legend{TCP,UDP,DNS,HTTP}
    \end{axis}
  \end{tikzpicture}
  \caption{Répartition des principaux protocoles par fichier.}
\end{figure}

\section{Top Talkers —}
\begin{figure}[H]
\centering
\begin{tikzpicture}
  \begin{axis}[
    ybar,
    width=0.9\textwidth,
    height=10cm,
    bar width=20pt,
    xlabel={Adresse IP},
    ylabel={Paquets},
    symbolic x coords={192.168.1.10,192.168.1.1,93.184.216.34,172.217.14.78,192.168.1.50},
    xtick=data,
    nodes near coords,
    nodes near coords align={vertical},
  ]
    \addplot+[ybar] coordinates {
      (192.168.1.10,3200)
      (192.168.1.1,2100)
      (93.184.216.34,1200)
      (172.217.14.78,900)
      (192.168.1.50,700)
    };
  \end{axis}
\end{tikzpicture}
\caption{Top 5 \textit{talkers} — Ethernet.}
\end{figure}

\chapter{Comparaison entre interfaces}
\section{Synthèse}
\begin{itemize}
  \item \textbf{Loopback} : trafic essentiellement applicatif local (diagnostic local, tests d'API). Peu de paquets mais échanges rapides.
  \item \textbf{Ethernet} : trafic général vers Internet et réseau local ; HTTP/HTTPS majoritaires ; présence de traffic « heavy-hitters ».
  \item \textbf{Wi-Fi} : mélange de trames de management + données, retransmissions plus fréquentes, chiffrement côté application (HTTPS).
\end{itemize}

\section{Observations de sécurité}
\begin{itemize}
  \item Requêtes DNS fréquentes — vérifier requêtes inhabituelles (fuzzing, exfiltration via DNS).
  \item Présence d'ARP peut indiquer découverte réseau normale ; vérifier ARP spoofing si duplication d'adresses MAC/IP.
  \item Absence de traffics classiques ou forte proportion de paquets ICMP/UDP inhabituels peut indiquer scans ou attaques.
\end{itemize}

\chapter{Conclusions}
\section{Conclusions}
L'examen des captures montre des comportements courants : connexions HTTP/HTTPS, DNS, échanges locaux sur loopback. Le Wi-Fi montre des signes de retransmission et plus de trames de management, attendus dans un réseau sans fil.


\chapter*{Annexes}
\addcontentsline{toc}{chapter}{Annexes}




\section*{1. Filtres Wireshark utiles}
\begin{itemize}
  \item Filtrer DNS : \texttt{dns}
  \item Filtrer HTTP : \texttt{http}
  \item Filtrer HTTPS (SNI / TLS) : \texttt{tls}
  \item Filtrer un IP source : \texttt{ip.src == 192.168.1.10}
  \item Filtrer retransmission TCP : \texttt{tcp.analysis.retransmission}
\end{itemize}  \vspace{4cm}

\chapter*{A propos de l'auteur}
Ce rapport a été préparé pour le projet d'analyse réseau.  \\ 
\textbf{\large Auteur : Yacine Sehli}

\vspace{1cm}
\noindent\textit{Fin du rapport}

\end{document}
